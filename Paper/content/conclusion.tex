\section{Conclusion}
Overall, the results seem to suggest that changing relations of an agent influence the distinctness of the trajectories it has.
The fact that the clusters are still closer together proves that there is potential to create distinctness but that it lacks coordination.

Coordination and self-organization is more easily achieved if the behaviour of the agents are a bit different or that they have more knowledge of the whole population, as is the case in the work of \cite{de2010multi}.

Giving agents relationships resembles real-life situations a bit more but provides an obstacle that slows down the goal of the population.
The number of agents clearly influence the overall speed in creating distinct clusters in which trajectories reside.
This also shows that trajectories take way longer and might not fully diverge from each other if the population is too large.

Social networks and especially changing networks present a new an interesting way to look at the evolution of speech.
But this way of thinking is not limited to speech.
It can be used in other scenarios where agent-based models are used to mimic social creatures.