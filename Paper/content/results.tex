\section{Results}
The simulations follow a similar pattern to what \cite{de2010multi} did.
First a social network of new agents is created and than the simulation is started.
In every iteration a random agent is select from the network and shifts a random trajectory.
It then plays 100 \textit{Imitation Games} with every friend it has, to get a success value of this shifted trajectory.
After these games, an accept or reject is done based on the number of successful \textit{Imitation Games}.
If the trajectory is accepted, a mix between the original trajectory and the shifted trajectory is made and stored in the agent (the success value for this trajectory is also updated as seen in \citep{de2010multi}).
After this is done, the agent has a \textit{falling out} with one of its friends (if it has more then two) based on the success of the \textit{Imitation Game} the two agents played.
Finally, the agent is presented with a new friend via the FoaF method where a friend of the agent, suggest a new friend.
And then the next iteration starts.

\begin{table}[t]
    \centering
    \begin{tabular}{ll}
    \hline
    Parameter                                         & Value  \\ \hline
    Number of Iterations                              & 600000 \\
    Number of games                                   & 100    \\
    $\sigma_{noise}$                                  & 2.0    \\
    $\sigma_{shift}$                                  & 1.0    \\
    $\beta$ (mix-factor)                              & 0.5    \\
    Number of agents                                  & 100    \\
    Number of trajectories                            & 4      \\
    Trajectory length                                 & 20     \\
    Maximum distance between neighboring points ($R$) & 1.0    \\
    Space size                                        & 10     \\
    Average number of friends                         & 10    
\end{tabular}
    \caption{Table to test captions and labels.}
    \label{table:simulation}
\end{table}

The results presented in the paper where achieved by running the simulation with the values found in \autoref{table:simulation}.
Most of the values are identical to the simulation parameters found in \citep*{de2010multi}.

% todo create plots and place them in the paper

% todo write rest of paper