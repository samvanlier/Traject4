\section{The Model}
The model that is used in this paper is very similar to the model described in \citep*{de2010multi}.
To recap, the model consists of a population of agents.
Each agents has four trajectories, with each trajectory having 20 points each.
These points have to adhere to a few constraints, first of all, they are bound in a space of size 10 units.
Secondly, neighboring points on a trajectory can have a maximum euclidean distance of 1 unit between them.

The goal of the model is to shift these trajectories so that acoustic distinction is achieved.
This paper aims to do this as wel, so almost all other functions, methods and parameters are borrowed from \cite*{de2010multi};

\subsection{The Social Network}
As mentioned before, this model is based on the model of \citeauthor{de2010multi}, but introduces social dynamics.
The original work used a population of 10 agents, which is never enough for a social network.
This resulted in the choice of increasing the number of agents to 100.

The social network that is used is a random network, which is made using the \textbf{Erdős–Rényi model} \citep*{erdos1959r}.
This model has been tweaked to make sure all nodes in the network are connected.
This is done before the \textbf{Erdős–Rényi model} starts working.
This has been done to ensure that every agent has at least one friend.

To make sure agents have enough opportunities to play the \textit{Imitation Game}, the number of edges has been chosen to be 500.
This is the number of edges needed to have an overall average degree of 10 in the network.
This can be checked using the Handshake-lemma of Euler.
\[
    \sum_{v \in V} deg(v) = 2 |E|
\]
with $V$ being the nodes in a graph and $E$ the edges of the graph.
$deg(v)$ is the degree of a node, also known as the number of connected edges to that node.
The average degree can be written as:
\[
    \frac{
        \sum_{v \in V} deg(v)
    }{|V|} = 10
    \Leftrightarrow
    \sum_{v \in V} deg(v) = 10 * |V|
\]
Now we substitute $\sum_{v \in V} deg(v)$ with $2|E|$ and solve the equation, resulting in:
\[    
    |E| = 5 * |V|
\]

The average degree of 10 was chosen so that the overall number of games per iteration are similar to the original work.

\subsection{Changing The Social Network}
During a simulation, the relationships between the agents can change.
Friends come and go, and can even reconnect, creating a network that is different every iteration of the simulation.

The relations can be changed in two ways, \textit{Friend of a Friend} (FoaF for short) and \textit{having a falling out}.
FoaF is term borrowed from ontologies but it fits for what this does.
\textit{Friend of a Friend} in this context means the introduction of a new friend via an already connected friend.
To place this is a real-life context, you go to a house party of our friend and he/she introduces you to a new person, this person is a \textit{Friend of a Friend}.
But agents can also have a \textit{falling out}.
In more concrete terms, they have a fight/argument and don't want to talk to each other for a time.
The agents in question disappear from each other friend list and cannot talk to each other anymore.
How both method work are explained in the following two subsections.

Both methods can produce unwanted results.
FoaF can lead to a fully connected network, where everyone talks to everyone else and \textit{having a falling out} can lead to lonely agents with no friends.
So this has to be countered and this can be done by doing both, but with some restrictions.
If an agents has less then 2 friends, no friend is removed.
This makes sure that no lonely agents exists.
Also, agents cannot suggest friends that are already in the friend list of the other agent, it is not a \textit{Friend of a Friend} if an agent is already friends with them.
So this could lead to a situation where there are new friends to suggest, so in this case the friend list does not grow.

\subsubsection{Friend of a Friend}
To select the FoaF, a friend needs to be picked that is suited to suggest a new friend to this agent.
This friend is chosen based on the following chance:
\begin{equation}
    P(friend) = \frac{|Friend.Friends|}{PotFoaFs(a)}
\end{equation}
$PotFoaFs(a)$ is defined as
\[
    PotFoaFs(a) = \sum_{f \in a.Friends} |f.Friends|
\]
which are all potential FoaF candidates (including the mutual friends).

This gives agents that are more social (e.i. have more friends) a higher change of being select to suggest a friend.
The suggested FoaF friend, however, is a random agent from the friend-list of the select friend.

\subsubsection{Having a falling out}
Agents can have a falling out, which means they are removed from each others friend-lists.
\textit{Falling outs} are based on the communication success between these agents.
In a single iteration of the simulation, an agent plays the \textit{Imitation game} with every friend it has for a number of times.
A success counter is kept to see what the success of the game was with every friend.
When a friends needs to be pick to remove from the friend-list, it is done using the following probability:
\begin{equation}
    P(friend) = \frac{friend.Succes}{OverallSuccess}
\end{equation}
$OverallSuccess$ is the total number of successful \textit{Imitation games} in the iteration.

If $OverallSuccess = 0$ then a random friend is picked to remove from the friend-list.
As mentioned before, this is not done when the selected friend or current agent has less than two friends.