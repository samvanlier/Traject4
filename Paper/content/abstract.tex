\begin{abstract}
    Humans are known to be social creatures.
    Most of our lives are driven by social interactions and communication.
    From our first day in the world, we are surrounded by other people with which we interact.
    But the social network of a person can grow and change over time.
    People interact, meet new people or lose connections with them which impacts the language a person speaks.
    This paper explores the impact of these changing social dynamics and how it impacts the evolution of a language when agents in a population do not have direct contact with every other agent but has a select group of ``friends'' with which an agent can talk.
    % todo add preview of the results
\end{abstract}